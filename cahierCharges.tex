\documentclass[majeure,gl]{tb}

\usepackage{longtable}
\usepackage{colortbl}
\usepackage{version}
\hypersetup{hidelinks=true}
\excludeversion{solution}
\includeversion{indication}

\date{\today} 
\numero{} 
\title[Cahier des charges]{Cahier des charges} 
\subtitle{Projet DEV \\ Révision : 1}
\author{MOA : Rémy HUBSCHER, Nicolas JULLIEN \\ MOE : Henri AYNIÉ, Yanni CUI, Thomas ROKICKI \\ \small(Groupe 9)}


\begin{document}

\maketitle

\clearpage
\renewcommand\contentsname{Sommaire}
\tableofcontents

\clearpage

\clearpage
\section{Introduction}

\subsection{Objet du document}

Ce document est destiné à expliciter et formaliser le besoin du client
dans le cadre du projet DEV qui se déroule pendant le 1er semestre de
2017.


\subsection{Portée du document}

Ce document décrit le projet logiciel de gestion de fuseaux horaires,
son plan d'action du prototype au produit finit, ainsi que le planning
opérationel du projet.

\subsection{Abréviations}
\label{sec:abreviation}

\begin{tabular}[c]{|p{1cm}|p{2.6cm}|p{12.4cm}|}
  \hline
  \textbf{Abr.} & \textbf{Signification} & \textbf{Libellé}
  \\\hline
  \textit{VIT} & \textit{Vitale} & \textit{Exigence fonctionnelle ou non fonctionnelle indispensable}
  \\\hline
  \textit{IMP} & \textit{Importante} & \textit{Exigence souhaitée mais non exigée}
  \\\hline
  \textit{MIN} & \textit{Mineure} &
  \textit{Exigence non exigée immédiatement, mais qui devra être prise en compte
    ultérieurement par le produit (impact sur l'évolutivité)}
  \\\hline
  \textit{CNIL} & \textit{Commission Nationale de l'Informatique et des Libertés} &
  \textit{Institution indépendante chargée de veiller au respect de l'identité humaine, de la vie privée et des libertées dans le monde numérique en France}
  \\\hline
  \textit{GPL} & \textit{Licence Publique Générale GNU} &
  \textit{Licence qui fixe les conditions légales de distribution des logiciels libres du projet GNU}
  \\\hline
\end{tabular}

\section{Les objectifs du produit}

\subsection{Définition du produit}

Le produit demandé par le client est une solution logicielle
facilitant la prise de rendez-vous pour des équipes distribuées sur
plusieurs fuseaux horaires.

\subsection{Contexte d'exploitation du produit}
Le produit sera accessible par les utilisateurs à partir d'un
ordinateur de type PC classique à l'aide d'une interface web.

Le produit sera accessible par les utilisateurs à partir de tout
dispositif d'accès au web :

\begin{itemize}
  \item Ordinateur de bureau
  \item Tablette
  \item Smartphone
  \item TV
\end{itemize}


\section{Exigences sur le produit}
\label{sec:exigence}

\subsection{Capacités Fonctionnelles}
\label{fonc}

\subsubsection{Description des fonctionnalités}

\paragraph{Informations utilisateur.} Les fonctions suivantes servent à obtenir les informations sur l'utilisateur

\begin{table}[ht]
  \centering
  \begin{tabular}[c]{|p{3cm}|p{13cm}|}
 \hline
 \textbf{Nom} &
 \textbf{Entrer position utilisateur}
 \\\hline
 \textbf{Description} &
 L'utilisateur entre sa position dans le système
 \\\hline
 \textbf{Évènement déclencheur} &
 Arrivée de l'utilisateur dans le système OU demande explicite de l'utilisateur
 \\\hline
 
 \textbf{Entrées} &
 Position de l'utilisateur dans une liste
 \\\hline
 \textbf{Sorties} &
 Erreur si contraintes non respectées.
 \\\hline
 \textbf{Contraintes} &
 La localisation doit être connue du site
 \\\hline
 \textbf{Importance} &
 VIT
 \\\hline 
 
 \hline
 \textbf{Nom} &
 \textbf{Entrer fuseau utilisateur}
 \\\hline
 \textbf{Description} &
 L'utilisateur entre son fuseau horaire
 \\\hline
 \textbf{Évènement déclencheur} &
 Arrivée de l'utilisateur dans le système OU demande explicite de l'utilisateur
 \\\hline
 
 \textbf{Entrées} &
 Selection du fuseau dans une liste.
 \\\hline
 \textbf{Sorties} &
 Erreur si contraintes non respectées.
 \\\hline
 \textbf{Contraintes} &
 Le fuseau doit exister
 \\\hline
 \textbf{Importance} &
 VIT
 \\\hline 

          \end{tabular}
\end{table}


 \paragraph{Consultation des horaires dans d'autres fuseaux} Les fonctions suivantes servent à permettre à l'utilisateur de voir l'heure qui l'intéresse dans d'autres fuseaux.
 
 \begin{table}[ht]
  \centering
  \begin{tabular}[c]{|p{3cm}|p{13cm}|}

\hline
 \textbf{Nom} &
 \textbf{Entrer fuseau partenaire}
 \\\hline
 \textbf{Description} &
 L'utilisateur entre le fuseau ou la position du partenaire
 \\\hline
 \textbf{Évènement déclencheur} &
 Demande explicite de l'utilisateur
 \\\hline
 
 \textbf{Entrées} &
 Position ou fuseau
 \\\hline
 \textbf{Sorties} &
 Erreur si contraintes non respectées.
 \\\hline
 \textbf{Contraintes} &
 La localisation doit être connue du site
 \\\hline
 \textbf{Importance} &
 VIT
 \\\hline 
 
 \hline
 \textbf{Nom} &
 \textbf{Consulter heure}
 \\\hline
 \textbf{Description} &
 L'utilisateur entre l'heure qu'il veut voir dans les différents fuseaux et le système l'affiche
 \\\hline
 \textbf{Évènement déclencheur} &
 Demande explicite de l'utilisateur
 \\\hline
 
 \textbf{Entrées} &
 Une heure
 \\\hline
 \textbf{Sorties} &
 L'heure sélectionnée dans les différents fuseaux
 \\\hline
 \textbf{Contraintes} &
 L'heure doit exister
 \\\hline
 \textbf{Importance} &
 VIT
 \\\hline 

    
          \end{tabular}
\end{table}


\clearpage

\paragraph{Version améliorée} Les fonctions suivantes sont utilisées pour une version plus développée du logiciel.

 \begin{table}[ht]
  \centering
  \begin{tabular}[c]{|p{3cm}|p{13cm}|}

\hline
 \textbf{Nom} &
 \textbf{Créer partenaire}
 \\\hline
 \textbf{Description} &
 L'utilisateur entre le fuseau ou la position du partenaire qui sera retenue par le système entre différentes connexions.
 \\\hline
 \textbf{Évènement déclencheur} &
 Demande explicite de l'utilisateur
 \\\hline
 
 \textbf{Entrées} &
 Position ou fuseau et nom du partenaire
 \\\hline
 \textbf{Sorties} &
 Erreur si contraintes non respectées.
 \\\hline
 \textbf{Contraintes} &
 La localisation doit être connue du site, toutes les entrées respectées
 \\\hline
 \textbf{Importance} &
 IMP
 \\\hline 
 
 \hline
 \textbf{Nom} &
 \textbf{Consulter heure avec partenaire}
 \\\hline
 \textbf{Description} &
 L'utilisateur entre l'heure qu'il veut voir et le système l'affiche dans les fuseaux des différents partenaires
 \\\hline
 \textbf{Évènement déclencheur} &
 Demande explicite de l'utilisateur
 \\\hline
 
 \textbf{Entrées} &
 Une heure
 \\\hline
 \textbf{Sorties} &
 L'heure sélectionnée dans les différents fuseaux
 \\\hline
 \textbf{Contraintes} &
 L'heure doit exister
 \\\hline
 \textbf{Importance} &
 IMP
 \\\hline 

    
          \end{tabular}
\end{table}


\subsubsection{Interopérabilité}
Le prototype logiciel sera hébergé sur le web. Il sera accessible depuis au moins un navigateur.

\subsection{Exigences non fonctionnelles}
\label{nonfonc}

\subsubsection{Rendement}
Le temps de réponse maximum autorisé du site sera de ~250 ms afin que l’utilisateur ne voit pas le retard entre les actions.

\subsubsection{Facilité d'utilisation}
Un utilisateur qui ne connaît pas le système pourra exploiter le système en au plus 30 minutes.


\subsubsection{Maintenabilité}
Le code respectera les conventions de nommages usuelles et sera rédigé et commenté en Anglais


\subsubsection{Portabilité}
Le prototype fonctionnera sur n'importe quel système d'exploitation et à au moins un navigateur.

\subsection{Exigences concernant le développement du produit}
\label{exigences}

\subsubsection{Objectifs de délais}
Le produit ainsi qu’un rapport et un poster sont à terminer avant la fin du projet en Mai 2017 \\
Des pénalités seront appliquées en cas de retard.

\subsubsection{Objectifs de coûts}
Tous les outils utilisés pour la réalisation du produitétant libres et gratuits, le projet n’aura pas de coût financier. \\
L’effort estimé est de 108h par élèves.

\subsubsection{Exigences de réalisation}
Le prototype sera programmé dans le langage Elm.
 Toutes les informations ajoutées dans le code
source le seront en anglais afin de faciliter l'intégration
d'ingénieurs étrangers.

Le code sera livré sous licence GPL, la documentation le sera sous
licence Creative Commons CC-by-sa.


\section{Synthèse des exigences}

\subsection{Hiérarchisation des exigences fonctionnelles} 
\label{sec:hiera}





\begin{center}
\begin{tabular}[c]{|p{10cm}|l|}
\hline
\textbf{Nom fonction} & \textbf{Importance}\\
\hline
Entrer position utilisateur
& VIT\\
\hline
Entrer fuseau utilisateur
& VIT\\
\hline
Entrer fuseau partenaire
& VIT\\
\hline
Consulter heure
& VIT\\
\hline
\hline
Créer partenaire
& IMP\\
\hline
Consulter heure avec partenaire
& IMP\\
\hline



\end{tabular}
\end{center}

\subsection{Hiérarchisation des exigences non-fonctionnelles} 


\begin{center}
\begin{tabular}[c]{|p{10cm}|l|}
\hline
\textbf{Exigence} & \textbf{Importance}\\
\hline
Maintenabilité
&VIT\\
\hline
Facilité d'utilisation
&VIT\\
\hline
Portabilité
&IMP\\
\hline
Rendement
&MIN\\
\hline
\end{tabular}
\end{center}

\end{document}
