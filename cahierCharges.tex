\documentclass[majeure,gl]{tb}

\usepackage{longtable}
\usepackage{colortbl}
\usepackage{version}
\hypersetup{hidelinks=true}
\excludeversion{solution}
\includeversion{indication}

\date{\today} 
\numero{} 
\title[Cahier des charges]{Cahier des charges} 
\subtitle{Projet DEV \\ Révision : 1}
\author{MOA : Rémy HUBSCHER, Nicolas JULLIEN \\ MOE : Henri AYNIÉ, Yanni CUI, Thomas ROKICKI \\ \small(Groupe 9)}


\begin{document}

\maketitle

\clearpage
\renewcommand\contentsname{Sommaire}
\tableofcontents


\clearpage






\clearpage
\section{Introduction}

\subsection{Objet du document}


\subsection{Portée du document}


\subsection{Abréviations}

\section{Les objectifs du produit}

\subsection{Définition du produit}

\subsection{Contexte d'exploitation du produit}

\section{Exigences sur le produit}
\label{sec:exigence}

\subsection{Capacités Fonctionnelles}
\label{fonc}

\subsubsection{Description des fonctionnalités}


\subsubsection{Interopérabilité}

\subsection{Exigences non fonctionnelles}
\label{nonfonc}

\subsubsection{Rendement}

\subsubsection{Facilité d'utilisation}


\subsubsection{Maintenabilité}


\subsubsection{Portabilité}

\subsection{Exigences concernant le développement du produit}
\label{exigences}

\subsubsection{Objectifs de délais}

\subsubsection{Objectifs de coûts}

\subsubsection{Exigences de réalisation}

\section{Synthèse des exigences}

\subsection{Hiérarchisation des exigences fonctionnelles} 
\label{sec:hiera}


\subsection{Hiérarchisation des exigences non-fonctionnelles} 


\end{document}
